\documentclass[11pt]{article}
\usepackage[a4paper,left=1.5cm,right=1.5cm,top=1.5cm,bottom=1.5cm]{geometry}
\usepackage{fancyhdr}
\renewcommand{\headrulewidth}{1pt}
\fancyhead[C]{\textbf{[LINMA1731] Second Homework}}
\fancyhead[L]{March 2019}
\fancyhead[R]{Gilles Peiffer [24321600]}

\usepackage{tikz}
\usepackage{pgfplots}
\usepackage[T1]{fontenc}
\usepackage[utf8]{inputenc}
\usepackage[english]{babel}
\usepackage{graphicx}
\usepackage{subcaption}
\usepackage{csquotes}
\usepackage{mathtools,amssymb}
\usepackage[binary-units=true,separate-uncertainty = true,multi-part-units=single]{siunitx}
\usepackage{float}
\usepackage[linktoc=all]{hyperref}
\hypersetup{breaklinks=true}
\newcommand{\hsp}{\hspace{20pt}}
\newcommand{\HRule}{\rule{\linewidth}{0.5mm}}
\graphicspath{{img/}}
\usepackage{caption}
\usepackage{textcomp}
\usepackage{array}
\usepackage{color}
\usepackage{tabularx,booktabs}
\usepackage{titlesec}
\usepackage{wrapfig}
\pagestyle{fancy}
\DeclarePairedDelimiterX{\norm}[1]{\lVert}{\rVert}{#1}

\DeclareSymbolFont{sfletters}{OML}{cmbrm}{m}{it}

\DeclareMathSymbol{\smu}{\mathord}{sfletters}{"16}
\DeclareMathSymbol{\ssigma}{\mathord}{sfletters}{"1B}
\DeclareMathSymbol{\sphi}{\mathord}{sfletters}{"1E}
\newcommand{\imag}{\mathrm{i}\mkern1mu} % Imaginary unit
\newcommand{\imagj}{\mathrm{j}\mkern1mu} % Imaginary unit but with 100% more engineering
\newcommand{\abs}[1]{\left\lvert#1\right\lvert}
\usepackage{listings}
\lstset{
	language=Python,
	numbers=left,
	numberstyle=\tiny\color{gray},
	basicstyle=\rm\small\ttfamily,
	keywordstyle=\bfseries\color{dkred},
	frame=single,
	commentstyle=\color{gray}=small,
	stringstyle=\color{dkgreen},
	%backgroundcolor=\color{gray!10},
	%tabsize=8, % Thank you Papa Torvalds
	%rulecolor=\color{black!30},
	%title=\lstname,
	breaklines=true,
	framextopmargin=2pt,
	framexbottommargin=2pt,
	extendedchars=true,
	inputencoding=utf8,
}

\DeclareMathOperator{\newdiff}{d} % use \dif instead
\newcommand{\dif}{\newdiff\!}

\newcommand\givenbase[1][]{\nonscript\:#1\lvert\nonscript\:}
\let\given\givenbase
\newcommand\sgiven{\givenbase[\delimsize]}
\DeclarePairedDelimiterX\Basics[1](){\let\given\sgiven #1}

\makeatletter
\DeclareRobustCommand{\Pr}{\operatorname{Pr}\@ifstar\@firstofone\@Pr}
\newcommand{\@Pr}[1]{\left\{#1\right\}}
\makeatother

\newcommand{\cdf}{\mathsf{F}}
\newcommand{\pdf}{\mathsf{T}}
\newcommand{\momnc}{\mathsf{m}}
\newcommand{\mom}{\smu}
\newcommand{\sd}{\ssigma}
\newcommand{\vars}{\ssigma^2}
\newcommand{\charfun}{\sphi}
\newcommand{\vecX}{\bm{X}}
\newcommand{\vecXc}{\vecX_{\textnormal{c}}}
\newcommand{\Hc}{H_\textnormal{c}}
\newcommand{\vecY}{\bm{Y}}
\newcommand{\vecYc}{\vecY_{\textnormal{c}}}
\newcommand{\vecZ}{\bm{Z}}
\newcommand{\vecZc}{\vecZ_{\textnormal{c}}}
\newcommand{\vecmom}{\bm{\momnc}}
\newcommand{\cov}{\mathsf{C}}
\newcommand{\covmat}{\bm{\cov}}
\newcommand{\Amat}{\bm{A}}
\newcommand{\hilb}{\textnormal{H}}
\makeatletter
\DeclareRobustCommand{\expe}{\mathsf{E}\@ifstar\@firstofone\@expe}
\newcommand{\@expe}[1]{\left[#1\right]}
\makeatother

\makeatletter
\DeclareRobustCommand{\var}{\mathsf{Var}\@ifstar\@firstofone\@var}
\newcommand{\@var}[1]{\left[#1\right]}
\makeatother

\begin{document}
\section{Statement}
Let $X(t)$ be a WSS stochastic process
with autocovariance function $\cov_X(\tau)$.
Let us consider the two following continuous-time stochastic processes,
\begin{equation}
Y(t) = X(t) \cos(\omega t + \phi) \quad \textnormal{and} \quad Z(t) = X(t) \cos[(\omega + \delta)t + \phi]\,,
\end{equation}
where $\omega$ and $\delta$ are constant and $\phi$ is a random variable independent of $X(t)$ and uniformly distributed on $[0, 2\pi]$.
Their autocovariance functions are
\begin{itemize}
	\item $\cov_Y(\tau) = \frac{1}{2} \cov_X(\tau) \cos(\omega \tau)$;
	\item $\cov_Z(\tau) = \frac{1}{2} \cov_X(\tau) \cos[(\omega + \delta) \tau]$.
\end{itemize}
Discuss the wide sense stationarity of process $H(t) = Y(t) + Z(t)$.

\section{Solution}
In order for $H(t)$ to be WSS, the signal should respect two conditions:
\begin{itemize}
	\item The mean should not depend on the time.
	\item The covariance between two times $t$ and $t'$ should only depend on their difference, $\tau = t - t'$.
\end{itemize}

\subsection{Condition on the mean}
We thus compute
\begin{align}
\momnc_H(t) &= \expe{H(t)} \\
&= \expe{X(t)[\cos(\omega t + \phi) + \cos((\omega+\delta)t + \phi)]} \\
&= \expe{X(t)}\expe{\cos(\omega t + \phi) + \cos((\omega+\delta)t + \phi)} \\
&= \frac{\momnc_X}{2\pi}\int_{0}^{2\pi} [\cos(\omega t + \phi) + \cos((\omega+\delta)t + \phi)]\dif \phi \\
&= \frac{\momnc_X}{2\pi} \left(\int_{\omega t}^{2\pi+ \omega t} \cos \phi' \dif \phi'+ \int_{(\omega + \delta) t}^{2\pi + (\omega + \delta) t} \cos \phi' \dif \phi' \right) \\
&= 0\,.
\end{align}
Hence the mean of $Y(t)$, $Z(t)$ and $H(t)$ does not depend on $t$.

\subsection{Condition on the autocovariance}
Next, we look at the autocovariance.
We compute (using the previous result)
\begin{align}
\cov_{H}(t, t') &= \expe{\Hc(t) \Hc^\hilb(t')} \\
\intertext{where ${}^\hilb$ denotes the transjugate matrix,}
&= \expe{H(t)H^\hilb(t')} \\
&= \expe{(Y(t) + Z(t))(Y^\hilb(t') + Z^\hilb(t'))}\,, \\
&= \expe{Y(t)Y^\hilb(t')} + \expe{Z(t)Z^\hilb(t')} + \expe{Y(t)Z^\hilb(t')} + \expe{Z(t)Y^\hilb(t')} \\
&= \cov_Y(\tau) + \cov_Z(\tau) + 2\expe{X(t)X^\hilb(t') \cos(\omega t + \phi) \cos((\omega + \delta)t' + \phi)}\\
&=  \cov_Y(\tau) + \cov_Z(\tau) + 2 \left(\cov_X(\tau) + \momnc_X \momnc_X^\hilb\right)\expe{\cos(\omega t + \phi) \cos((\omega + \delta)t' + \phi)}\,.
\end{align}
We then only have to determine whether the expectation in that last term depends only on $\tau$.
\begin{align}
\expe{\cos(\omega t + \phi) \cos((\omega + \delta)t' + \phi)} &= \expe{\cos(\omega\tau - \delta t') + \cos(\omega (t + t') + \delta t' + 2 \phi)}\\
&= \cos(\omega\tau - \delta t') + \frac{1}{2\pi} \int_0^{2\pi} \cos(\omega (t + t') + \delta t' + 2 \phi) \dif \phi \\
&= \cos(\omega \tau - \delta t')\,.
\end{align}
This final term should only depend on $\tau$ in order for the autocovariance to only depend on $\tau$, and hence make $H(t)$ a WSS stochastic process.
The only way for this to be true, is to have $\delta = 0$.
Since there was no previous constraint on $H(t)$ because of the mean, we can conclude that \fbox{$H(t)$ is a WSS process if and only if $\delta = 0$.}
\end{document}