\documentclass[11pt]{article}
\usepackage[a4paper,left=1.5cm,right=1.5cm,top=1.5cm,bottom=1.5cm]{geometry}
\usepackage{fancyhdr}
\renewcommand{\headrulewidth}{1pt}
\fancyhead[C]{\textsc{[LINMA1731] Fourth Homework}}
\fancyhead[L]{18 March 2019}
\fancyhead[R]{Gilles \textsc{Peiffer} (24321600)}

\usepackage[dvipsnames]{xcolor}
\usepackage{tikz}
\usepackage{pgfplots}
\usepackage[T1]{fontenc}
\usepackage[utf8]{inputenc}
\usepackage[english]{babel}
\usepackage{graphicx}
\usepackage{subcaption}
\usepackage{csquotes}
\usepackage{mathtools,amssymb}
\usepackage[binary-units=true,separate-uncertainty = true,multi-part-units=single]{siunitx}
\usepackage{float}
\usepackage[linktoc=all]{hyperref}
\hypersetup{breaklinks=true}
\newcommand{\hsp}{\hspace{20pt}}
\newcommand{\HRule}{\rule{\linewidth}{0.5mm}}
\graphicspath{{img/}}
\usepackage{caption}
\usepackage{textcomp}
\usepackage{array}
\usepackage{tabularx,booktabs}
\usepackage{titlesec}
\usepackage{wrapfig}
\pagestyle{fancy}
\usepackage{empheq}
\usepackage{mathrsfs}
\DeclarePairedDelimiterX{\norm}[1]{\lVert}{\rVert}{#1}
\definecolor{wolframred}{HTML}{DC1100}
\definecolor{wolframorange}{HTML}{FF7D00}

\DeclareSymbolFont{sfletters}{OML}{cmbrm}{m}{it}

\DeclareMathSymbol{\smu}{\mathord}{sfletters}{"16}
\DeclareMathSymbol{\ssigma}{\mathord}{sfletters}{"1B}
\DeclareMathSymbol{\sphi}{\mathord}{sfletters}{"1E}
\newcommand{\imag}{\mathrm{i}\mkern1mu} % Imaginary unit
\newcommand{\imagj}{\mathrm{j}\mkern1mu} % Imaginary unit but with 100% more engineering
\newcommand{\abs}[1]{\left\lvert#1\right\lvert}
\usepackage{listings}
\lstset{
	language=Python,
	numbers=left,
	numberstyle=\tiny\color{gray},
	basicstyle=\rm\small\ttfamily,
	keywordstyle=\bfseries\color{dkred},
	frame=single,
	commentstyle=\color{gray}=small,
	stringstyle=\color{dkgreen},
	%backgroundcolor=\color{gray!10},
	%tabsize=8, % Thank you Papa Torvalds
	%rulecolor=\color{black!30},
	%title=\lstname,
	breaklines=true,
	framextopmargin=2pt,
	framexbottommargin=2pt,
	extendedchars=true,
	inputencoding=utf8,
}

\DeclareMathOperator{\newdiff}{d} % use \dif instead
\newcommand{\dif}{\newdiff\!}
\newcommand{\e}{\mathrm{e}}

\newcommand\givenbase[1][]{\nonscript\:#1\lvert\nonscript\:}
\let\given\givenbase
\newcommand\sgiven{\givenbase[\delimsize]}
\DeclarePairedDelimiterX\Basics[1](){\let\given\sgiven #1}

\makeatletter
\DeclareRobustCommand{\Pr}{\operatorname{Pr}\@ifstar\@firstofone\@Pr}
\newcommand{\@Pr}[1]{\left\{#1\right\}}
\makeatother

\newcommand{\cdf}{\mathsf{F}}
\newcommand{\pdf}{\mathsf{T}}
\newcommand{\momnc}{\mathsf{m}}
\newcommand{\mom}{\smu}
\newcommand{\sd}{\ssigma}
\newcommand{\vars}{\ssigma^2}
\newcommand{\charfun}{\sphi}
\newcommand{\vecX}{\bm{X}}
\newcommand{\vecXc}{\vecX_{\textnormal{c}}}
\newcommand{\Hc}{H_\textnormal{c}}
\newcommand{\vecY}{\bm{Y}}
\newcommand{\vecYc}{\vecY_{\textnormal{c}}}
\newcommand{\vecZ}{\bm{Z}}
\newcommand{\vecZc}{\vecZ_{\textnormal{c}}}
\newcommand{\vecmom}{\bm{\momnc}}
\newcommand{\cov}{C}
\newcommand{\covmat}{\bm{\cov}}
\newcommand{\Amat}{\bm{A}}
\newcommand{\hilb}{\textnormal{H}}
\makeatletter
\DeclareRobustCommand{\expe}{\mathsf{E}\@ifstar\@firstofone\@expe}
\newcommand{\@expe}[1]{\left[#1\right]}
\makeatother

\renewcommand{\Re}{\operatorname{Re}}
\renewcommand{\Im}{\operatorname{Im}}

\makeatletter
\DeclareRobustCommand{\var}{\mathsf{Var}\@ifstar\@firstofone\@var}
\newcommand{\@var}[1]{\left[#1\right]}
\makeatother

\newcommand*\widebox[1]{\boxed{#1}}

\begin{document}
\section{Statement}
Let \(E(t)\) be a white noise with autocovariance function given by \(\cov_E(\tau) = 4 \delta(\tau)\).
This process passes through an LTI system \(\mathcal{L}\) (also called \emph{innovation filter}) and we get the real stochastic process \(Y(t)\) as output.
The power spectral density of \(Y(t)\), denoted by \(\gamma_Y(\omega)\), is given by
\begin{equation}
\gamma_Y(\omega) = \frac{64 \omega^2 + 64}{4 \omega^4 + 65 \omega^2 + 16}\,.\notag
\end{equation}

\begin{enumerate}
	\item What is the power spectral density of the white noise, i.e., \(\gamma_E(\omega)\)?
	\item Give the expressions of \(\gamma_E(s)\) and \(\gamma_Y(s)\), using the appropriate change of variable.
	\item Compute the transfer function \(L(s)\) of the LTI system \(\mathcal{L}\).
	Make sure the filter is minimum-phase.
\end{enumerate}

\section{Solution}
\begin{enumerate}
	\item For the first exercise,
	we must recall that the power spectral density is defined as the Fourier transform of the autocovariance function.
	We thus have
	\begin{align}
	\gamma_E(\omega) &= \mathscr{F}_\tau \left[\cov_E(\tau) \right](\omega) \notag \\
	&= \mathscr{F}_\tau \left[4 \delta(\tau)\right](\omega) \notag \\
	&= \int_{-\infty}^{+\infty} 4\delta(\tau) \e^{-\imagj \omega \tau} \dif \tau \notag \,.\\
	\intertext{Computing this integral, we find that}
	\Aboxed{\gamma_E(\omega) &= 4\,.}
	\end{align}
	\item Our current power spectral density expressions are functions of the imaginary variable \(\imagj \omega\).
	In order to express them as functions of \(s = \sigma + \imagj \omega\),
	we must bear in mind that \(\omega^2 = -s^2\) and that \(\omega^4 = s^4\).
	With these conversions, we find that
	\begin{empheq}[box=\widebox]{align}
	\gamma_Y(s) &= \frac{-64s^2 + 64}{4 s^4 - 65 s^2 + 16}\,,\\
	\gamma_E(s) &= 4\,.
	\end{empheq}
	\item In order to obtain a causal, stable filter, we must verify that the Paley--Wiener condition is satisfied, otherwise no such solution can exist.
	Using \textcolor{wolframred}{Wolfram}\textcolor{wolframorange}{Alpha}, we find that 
	\begin{equation}
	\int_{-\infty}^{+\infty} \frac{\abs{\ln\left(\frac{64 \omega^2 + 64}{4 \omega^4 + 65 \omega^2 + 16}\right)}}{1 + \omega^2} \dif \omega \approx 2.90319 < +\infty\,,\notag
	\end{equation}
	which means the condition is satisfied.
	Since \(Y(t)\) is the output of an LTI system with transfer function \(L(s)\)
	to which we feed a white noise \(E(t)\),
	we know that
	\begin{equation}
	\gamma_Y(s) = L(s) L(-s) \gamma_E(s)\,.\notag
	\end{equation}
	We then apply the spectral factorization procedure.
	\begin{align}
	\gamma_Y(s) &= L(s) L(-s) \gamma_E(s)\notag\\
	&= \underbrace{\left(2 \frac{s+1}{(s+4)(s+1/2)}\right)}_{L(s)} \underbrace{\left(2 \frac{-s+1}{(-s+4)(-s+1/2)}\right)}_{L(-s)} \gamma_E(s)\,.\notag
	\end{align}
	We thus obtain the following expression for the transfer function of the innovation filter \(\mathcal{L}\):
	\begin{equation}
	\boxed{L(s) = 2 \frac{s+1}{(s+4)(s+1/2)}\,.}
	\end{equation}
	In order to verify whether this filter is minimum-phase,
	we must check that the poles and zeroes are all in the open left-half plane,
	that is, \(\Re(p_i) < 0\) and \(\Re(z_i) < 0\), for all \(i\).
	The poles are \(p_1 = -4, p_2 = -1/2\) and the only zero is \(z_1 = -1\).
	This implies that the filter is minimum-phase, as required.
\end{enumerate}
\end{document}