\documentclass[11pt]{article}
\usepackage[a4paper,left=1.5cm,right=1.5cm,top=1.5cm,bottom=1.5cm]{geometry}
\usepackage{fancyhdr}
\renewcommand{\headrulewidth}{1pt}
\fancyhead[C]{\textsc{[LINMA1731] Sixth Homework}}
\fancyhead[L]{23 April 2019}
\fancyhead[R]{Gilles \textsc{Peiffer} (24321600)}

\usepackage[dvipsnames]{xcolor}
\usepackage{tikz}
\usepackage{pgfplots}
\usepackage[T1]{fontenc}
\usepackage[utf8]{inputenc}
\usepackage[english]{babel}
\usepackage{graphicx}
\usepackage{subcaption}
\usepackage{csquotes}
\usepackage{mathtools,amssymb}
\usepackage[binary-units=true,separate-uncertainty = true,multi-part-units=single]{siunitx}
\usepackage{float}
\usepackage[linktoc=all]{hyperref}
\hypersetup{breaklinks=true}
\newcommand{\hsp}{\hspace{20pt}}
\newcommand{\HRule}{\rule{\linewidth}{0.5mm}}
\graphicspath{{img/}}
\usepackage{caption}
\usepackage{textcomp}
\usepackage{array}
\usepackage{tabularx,booktabs}
\usepackage{titlesec}
\usepackage{wrapfig}
\pagestyle{fancy}
\usepackage{empheq}
\usepackage{mathrsfs}
\DeclarePairedDelimiterX{\norm}[1]{\lVert}{\rVert}{#1}
\definecolor{wolframred}{HTML}{DC1100}
\definecolor{wolframorange}{HTML}{FF7D00}

\DeclareSymbolFont{sfletters}{OML}{cmbrm}{m}{it}

\DeclareMathSymbol{\smu}{\mathord}{sfletters}{"16}
\DeclareMathSymbol{\ssigma}{\mathord}{sfletters}{"1B}
\DeclareMathSymbol{\sphi}{\mathord}{sfletters}{"1E}
\newcommand{\imag}{\mathrm{i}\mkern1mu} % Imaginary unit
\newcommand{\imagj}{\mathrm{j}\mkern1mu} % Imaginary unit but with 100% more engineering
\newcommand{\abs}[1]{\left\lvert#1\right\lvert}
\usepackage{listings}
\lstset{
	language=Python,
	numbers=left,
	numberstyle=\tiny\color{gray},
	basicstyle=\rm\small\ttfamily,
	keywordstyle=\bfseries\color{dkred},
	frame=single,
	commentstyle=\color{gray}=small,
	stringstyle=\color{dkgreen},
	%backgroundcolor=\color{gray!10},
	%tabsize=8, % Thank you Papa Torvalds
	%rulecolor=\color{black!30},
	%title=\lstname,
	breaklines=true,
	framextopmargin=2pt,
	framexbottommargin=2pt,
	extendedchars=true,
	inputencoding=utf8,
}

\DeclareMathOperator{\newdiff}{d} % use \dif instead
\newcommand{\dif}{\newdiff\!}
\newcommand{\e}{\mathrm{e}}

\newcommand\givenbase[1][]{\nonscript\:#1\lvert\nonscript\:}
\let\given\givenbase
\newcommand\sgiven{\givenbase[\delimsize]}
\DeclarePairedDelimiterX\Basics[1](){\let\given\sgiven #1}

\makeatletter
\DeclareRobustCommand{\Pr}{\operatorname{Pr}\@ifstar\@firstofone\@Pr}
\newcommand{\@Pr}[1]{\left\{#1\right\}}
\makeatother

\newcommand{\cdf}{\mathsf{F}}
\newcommand{\pdf}{\mathsf{f}}
\newcommand{\momnc}{\mathsf{m}}
\newcommand{\mom}{\smu}
\newcommand{\sd}{\ssigma}
\newcommand{\vars}{\ssigma^2}
\newcommand{\charfun}{\sphi}
\newcommand{\vecX}{\bm{X}}
\newcommand{\vecXc}{\vecX_{\textnormal{c}}}
\newcommand{\Hc}{H_\textnormal{c}}
\newcommand{\vecY}{\bm{Y}}
\newcommand{\vecYc}{\vecY_{\textnormal{c}}}
\newcommand{\vecZ}{\bm{Z}}
\newcommand{\vecZc}{\vecZ_{\textnormal{c}}}
\newcommand{\vecmom}{\bm{\momnc}}
\newcommand{\cov}{C}
\newcommand{\covmat}{\bm{\cov}}
\newcommand{\Amat}{\bm{A}}
\newcommand{\hilb}{\textnormal{H}}
\newcommand{\hTheta}{\widehat{\Theta}}
\newcommand{\wopt}{w_{\textnormal{opt}}}

\makeatletter
\DeclareRobustCommand{\expe}{\mathbb{E}\@ifstar\@firstofone\@expe}
\newcommand{\@expe}[1]{\left[#1\right]}
\makeatother

\renewcommand{\Re}{\operatorname{Re}}
\renewcommand{\Im}{\operatorname{Im}}

\makeatletter
\DeclareRobustCommand{\var}{\mathbb{V}\@ifstar\@firstofone\@var}
\newcommand{\@var}[1]{\left[#1\right]}
\makeatother

\newcommand*\widebox[1]{\boxed{#1}}

\begin{document}
\section{Statement}
Let \(X(n)\) be a real WSS process whose correlation function is given by \(R_X(k) = \e^{-k^2}\).
This signal is amplified by a factor \(c \in \mathbb{R}\) and corrupted by an additive white noise \(N(n)\)
of variance \(\vars_N\).
We hence observe a signal \(Y(n)\) given by \(Y(n) = cX(n) + N(n)\).

One would like to design a Wiener filter of order two (i.e. with two coefficients) in order to optimally compute an estimate of \(X(n)\) based on observations \(Y(n)\).
\begin{itemize}
	\item Based on the orthogonality principle, give the linear system of equations to solve in order to find the expression of the two coefficients.
	\item Give the expressions of the two coefficients of the filter.
	\item Give the expression of the transfer function \(W(z)\) of the Wiener filter.
\end{itemize}

\section{Solution}
\begin{itemize}
	\item The orthogonality principle says that \(\expe{\Big(X(n) - \hat{X}(n)\Big)\Big(cX(n) + N(n) \Big)} = 0\).
	From this, one can deduce the Wiener--Hopf equations,
	given in matrix form by
	\begin{equation}
	R_Y w = R_{XY}\,.\notag
	\end{equation}
	One can compute these matrices\footnote{Using properties of the covariance which also apply to the correlation due to its quadratic nature, as well as using the fact that \(X\) and \(N\) are uncorrelated.}:
	\begin{align}
	R_Y(k) &= c^2 R_X(k) + R_N(k) = c^2 \e^{-k^2} + \vars_N \delta(k)\,,\notag\\
	R_{XY}(k) &= c^2 \e^{-k^2}\notag\,.
	\end{align}
	Knowing that the filter is of order two,
	one then finds the following Wiener--Hopf equations:
	\begin{align}
	\begin{pmatrix} R_Y(0) & R_Y(-1) \\ R_Y(1) & R_Y(0) \end{pmatrix}
	\begin{pmatrix} w(0) \\ w(1) \end{pmatrix}
	&=
	\begin{pmatrix} R_{XY}(0) \\ R_{XY}(1) \end{pmatrix}\notag\\
	\implies
	\Aboxed{{\begin{pmatrix} c^2 + \vars_N & c^2/\e \\ c^2/\e & c^2 + \vars_N \end{pmatrix}}
	{\begin{pmatrix} w(0) \\ w(1) \end{pmatrix}}
	&=
	{\begin{pmatrix} c^2 \\ c^2/\e \end{pmatrix}}\,.}\label{eq:wh}
	\end{align}
	\item Rearranging the system in \eqref{eq:wh} to isolate the optimal filter, one finds
	\begin{equation}
	\begin{pmatrix} \wopt(0) \\ \wopt(1) \end{pmatrix}
	=
	\begin{pmatrix} c^2 + \vars_N & c^2/\e \\ c^2/\e & c^2 + \vars_N \end{pmatrix}^{-1}
	\begin{pmatrix} c^2 \\ c^2/\e \end{pmatrix}\,.\notag
	\end{equation}
	Solving this equation yields
	\begin{align}
	\begin{pmatrix} \wopt(0) \\ \wopt(1) \end{pmatrix}
	&= \frac{\e}{(\e^2 - 1)c^4 + 2 \e^2 c^2 \vars_N + \e^2 \sigma_N^4}
	\begin{pmatrix} \e c^2 + \e \vars_N & -c^2 \\ -c^2 & \e c^2 + \e \vars_N \end{pmatrix} \begin{pmatrix} c^2 \\ c^2/\e \end{pmatrix}\,.\notag
	\intertext{This finally evaluates to}
	\Aboxed{{\begin{pmatrix} \wopt(0) \\ \wopt(1) \end{pmatrix}} &= \frac{\e c^2}{(\e^2 - 1)c^4 + 2 \e^2 c^2 \vars_N + \e^2 \sigma_N^4} {\begin{pmatrix} \e c^2 + \e \vars_N - c^2/\e \\ \vars_N \end{pmatrix}}\,.}\label{eq:wopt}
	\end{align}
	\item The transfer function \(W(z)\) of the filter can then be found to be
	\begin{equation}
	\boxed{W(z) = \frac{\e c^2}{(\e^2 - 1)c^4 + 2 \e^2 c^2 \vars_N + \e^2 \sigma_N^4} \Big(\e c^2 + \e \vars_N - c^2/\e + z^{-1} \vars_N \Big)\,,}
	\end{equation}
	from \eqref{eq:wopt}.
\end{itemize}
\end{document}