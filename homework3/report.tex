\documentclass[11pt]{article}
\usepackage[a4paper,left=1.5cm,right=1.5cm,top=1.5cm,bottom=1.5cm]{geometry}
\usepackage{fancyhdr}
\renewcommand{\headrulewidth}{1pt}
\fancyhead[C]{\textsc{[LINMA1731] Third Homework}}
\fancyhead[L]{11 March 2019}
\fancyhead[R]{Gilles \textsc{Peiffer} (24321600)}

\usepackage{tikz}
\usepackage{pgfplots}
\usepackage[T1]{fontenc}
\usepackage[utf8]{inputenc}
\usepackage[english]{babel}
\usepackage{graphicx}
\usepackage{subcaption}
\usepackage{csquotes}
\usepackage{mathtools,amssymb}
\usepackage[binary-units=true,separate-uncertainty = true,multi-part-units=single]{siunitx}
\usepackage{float}
\usepackage[linktoc=all]{hyperref}
\hypersetup{breaklinks=true}
\newcommand{\hsp}{\hspace{20pt}}
\newcommand{\HRule}{\rule{\linewidth}{0.5mm}}
\graphicspath{{img/}}
\usepackage{caption}
\usepackage{textcomp}
\usepackage{array}
\usepackage{color}
\usepackage{tabularx,booktabs}
\usepackage{titlesec}
\usepackage{wrapfig}
\pagestyle{fancy}
\usepackage{mathrsfs}
\DeclarePairedDelimiterX{\norm}[1]{\lVert}{\rVert}{#1}

\DeclareSymbolFont{sfletters}{OML}{cmbrm}{m}{it}

\DeclareMathSymbol{\smu}{\mathord}{sfletters}{"16}
\DeclareMathSymbol{\ssigma}{\mathord}{sfletters}{"1B}
\DeclareMathSymbol{\sphi}{\mathord}{sfletters}{"1E}
\newcommand{\imag}{\mathrm{i}\mkern1mu} % Imaginary unit
\newcommand{\imagj}{\mathrm{j}\mkern1mu} % Imaginary unit but with 100% more engineering
\newcommand{\abs}[1]{\left\lvert#1\right\lvert}
\usepackage{listings}
\lstset{
	language=Python,
	numbers=left,
	numberstyle=\tiny\color{gray},
	basicstyle=\rm\small\ttfamily,
	keywordstyle=\bfseries\color{dkred},
	frame=single,
	commentstyle=\color{gray}=small,
	stringstyle=\color{dkgreen},
	%backgroundcolor=\color{gray!10},
	%tabsize=8, % Thank you Papa Torvalds
	%rulecolor=\color{black!30},
	%title=\lstname,
	breaklines=true,
	framextopmargin=2pt,
	framexbottommargin=2pt,
	extendedchars=true,
	inputencoding=utf8,
}

\DeclareMathOperator{\newdiff}{d} % use \dif instead
\newcommand{\dif}{\newdiff\!}
\newcommand{\e}{\mathrm{e}}

\newcommand\givenbase[1][]{\nonscript\:#1\lvert\nonscript\:}
\let\given\givenbase
\newcommand\sgiven{\givenbase[\delimsize]}
\DeclarePairedDelimiterX\Basics[1](){\let\given\sgiven #1}

\makeatletter
\DeclareRobustCommand{\Pr}{\operatorname{Pr}\@ifstar\@firstofone\@Pr}
\newcommand{\@Pr}[1]{\left\{#1\right\}}
\makeatother

\newcommand{\cdf}{\mathsf{F}}
\newcommand{\pdf}{\mathsf{T}}
\newcommand{\momnc}{\mathsf{m}}
\newcommand{\mom}{\smu}
\newcommand{\sd}{\ssigma}
\newcommand{\vars}{\ssigma^2}
\newcommand{\charfun}{\sphi}
\newcommand{\vecX}{\bm{X}}
\newcommand{\vecXc}{\vecX_{\textnormal{c}}}
\newcommand{\Hc}{H_\textnormal{c}}
\newcommand{\vecY}{\bm{Y}}
\newcommand{\vecYc}{\vecY_{\textnormal{c}}}
\newcommand{\vecZ}{\bm{Z}}
\newcommand{\vecZc}{\vecZ_{\textnormal{c}}}
\newcommand{\vecmom}{\bm{\momnc}}
\newcommand{\cov}{C}
\newcommand{\covmat}{\bm{\cov}}
\newcommand{\Amat}{\bm{A}}
\newcommand{\hilb}{\textnormal{H}}
\makeatletter
\DeclareRobustCommand{\expe}{\mathsf{E}\@ifstar\@firstofone\@expe}
\newcommand{\@expe}[1]{\left[#1\right]}
\makeatother

\renewcommand{\Re}{\operatorname{Re}}
\renewcommand{\Im}{\operatorname{Im}}

\makeatletter
\DeclareRobustCommand{\var}{\mathsf{Var}\@ifstar\@firstofone\@var}
\newcommand{\@var}[1]{\left[#1\right]}
\makeatother

\begin{document}
\section{Statement}
Let $X(t)$ be a continuous WSS stochastic process.
The covariance function of $X(t)$ is given by $\cov_X(\tau) = \e^{-\tau^2/2}$ and its mean by $\momnc_X = 1$.

$X(t)$ is filtered by an LTI system of impulse response $h(t) = \e^{-3t} u(t)$ to produce a new process $Y(t)$.

\begin{enumerate}
	\item Derive the expression of the power spectral density of $X(t)$.
	\item Derive the mean of $Y(t)$.
	\item Derive the expression of the power spectral density of $Y(t)$.
\end{enumerate}

\section{Solution}
\begin{enumerate}
	\item For the first exercise,
	we must recall that the power spectral density is defined as the Fourier transform of the autocovariance function.
	We thus have
	\begin{align}
	\gamma_X(\omega) &= \mathscr{F}_\tau \left[\cov_X(\tau) \right](\omega) \notag \\
	&= \mathscr{F}_\tau \left[\e^{-\tau^2/2}\right](\omega) \notag \\
	&= \int_{-\infty}^{+\infty} \e^{-\tau^2/2} \e^{-\imagj \omega \tau} \dif \tau \notag \,.\\
	\intertext{Computing this integral, we find that}
	\Aboxed{\gamma_X(\omega) &= \sqrt{2 \pi}  \e^{-\omega^2 /2}\,.} \label{eq:psdx}
	\end{align}
	\item In order to compute the mean of $Y(t)$,
	we remember that the output of an LTI system that receives a WSS input is also WSS:
	\begin{align}
	\momnc_Y &= \expe{Y(t)} \notag \\
	&= \int_{-\infty}^{+\infty} y(\tau) \dif \tau \notag \\
	&= \momnc_X \int_{-\infty}^{+\infty} h(\tau) \dif \tau \notag \\
	&= \int_{-\infty}^{+\infty} \e^{-3\tau} u(\tau) \dif \tau \notag \\
	&= \int_{0}^{+\infty} \e^{-3\tau} \dif \tau \notag\,. \\
	\intertext{Computing this integral yields}
	\Aboxed{\momnc_Y &= \frac{1}{3}\,.}
	\end{align}
	\item In order to compute the power spectral density of $Y(t)$,
	we must recall the Wiener--Khinchin theorem: since $Y(t)$ is the output of an LTI system to which a WSS random process is fed,
	\begin{align}
	\gamma_Y(\omega) &= \abs{H(\omega)}^2 \gamma_X(\omega)\,. \label{eq:psdrel} \\
	\intertext{In order to use the result of (\ref{eq:psdx}), we must compute the Fourier transform of the impulse response, $H(\omega)$:}
	H(\omega) &= \mathscr{F}_t\left[ h(t) \right](\omega) \notag \\
	&= \mathscr{F}_t\left[ \e^{-3t} u(t) \right](\omega) \notag \\
	&= \int_{-\infty}^{+\infty} \e^{-3\tau} u(\tau) \e^{\imagj \omega \tau}\dif \tau \notag \\
	&= \int_0^{+\infty} \e^{-3\tau} \e^{\imagj \omega \tau} \dif \tau \notag \\
	&= \frac{1}{\imagj \omega + 3}\,. \label{eq:freqresp}\\
	\intertext{With the result in (\ref{eq:freqresp}), we can easily compute $\gamma_Y(\omega)$ using (\ref{eq:psdrel}) and (\ref{eq:psdx}):}
	\Aboxed{\gamma_Y(\omega) &=  \frac{\sqrt{2\pi}}{\omega^2 + 9} \e^{-\omega^2/2}\,.}
	\end{align}
\end{enumerate}

\end{document}