\documentclass[11pt]{article}
\usepackage[a4paper,left=1.5cm,right=1.5cm,top=1.5cm,bottom=1.5cm]{geometry}
\usepackage{fancyhdr}
\renewcommand{\headrulewidth}{1pt}
\fancyhead[C]{\textbf{[LINMA1731] First Homework}}
\fancyhead[L]{February 2019}
\fancyhead[R]{Gilles Peiffer [24321600]}

\usepackage{tikz}
\usepackage{pgfplots}
\usepackage[T1]{fontenc}
\usepackage[utf8]{inputenc}
\usepackage[english]{babel}
\usepackage{graphicx}
\usepackage{subcaption}
\usepackage{csquotes}
\usepackage{mathtools,amssymb}
\usepackage[binary-units=true,separate-uncertainty = true,multi-part-units=single]{siunitx}
\usepackage{float}
\usepackage[linktoc=all]{hyperref}
\hypersetup{breaklinks=true}
\setlength{\parskip}{1ex plus 0.5ex minus 0.2ex}
\newcommand{\hsp}{\hspace{20pt}}
\newcommand{\HRule}{\rule{\linewidth}{0.5mm}}
\graphicspath{{img/}}
\usepackage{caption}
\usepackage{textcomp}
\usepackage{array}
\usepackage{color}
\usepackage{tabularx,booktabs}
\usepackage{titlesec}
\usepackage{wrapfig}
\titlespacing{\section}{0pt}{\parskip}{-\parskip}
\titlespacing{\subsection}{0pt}{\parskip}{-\parskip}
\titlespacing{\subsubsection}{0pt}{\parskip}{-\parskip}
\pagestyle{fancy}
\DeclarePairedDelimiterX{\norm}[1]{\lVert}{\rVert}{#1}

\DeclareSymbolFont{sfletters}{OML}{cmbrm}{m}{it}

\DeclareMathSymbol{\smu}{\mathord}{sfletters}{"16}
\DeclareMathSymbol{\ssigma}{\mathord}{sfletters}{"1B}
\DeclareMathSymbol{\sphi}{\mathord}{sfletters}{"1E}
\newcommand{\imag}{\mathrm{i}\mkern1mu} % Imaginary unit
\newcommand{\imagj}{\mathrm{j}\mkern1mu} % Imaginary unit but with 100% more engineering
\newcommand{\abs}[1]{\left\lvert#1\right\lvert}
\usepackage{listings}
\lstset{
	language=Python,
	numbers=left,
	numberstyle=\tiny\color{gray},
	basicstyle=\rm\small\ttfamily,
	keywordstyle=\bfseries\color{dkred},
	frame=single,
	commentstyle=\color{gray}=small,
	stringstyle=\color{dkgreen},
	%backgroundcolor=\color{gray!10},
	%tabsize=8, % Thank you Papa Torvalds
	%rulecolor=\color{black!30},
	%title=\lstname,
	breaklines=true,
	framextopmargin=2pt,
	framexbottommargin=2pt,
	extendedchars=true,
	inputencoding=utf8,
}

\DeclareMathOperator{\newdiff}{d} % use \dif instead
\newcommand{\dif}{\newdiff\!}

\newcommand\givenbase[1][]{\nonscript\:#1\lvert\nonscript\:}
\let\given\givenbase
\newcommand\sgiven{\givenbase[\delimsize]}
\DeclarePairedDelimiterX\Basics[1](){\let\given\sgiven #1}

\makeatletter
\DeclareRobustCommand{\Pr}{\operatorname{Pr}\@ifstar\@firstofone\@Pr}
\newcommand{\@Pr}[1]{\left\{#1\right\}}
\makeatother

\newcommand{\cdf}{\mathsf{F}}
\newcommand{\pdf}{\mathsf{T}}
\newcommand{\momnc}{\mathsf{m}}
\newcommand{\mom}{\smu}
\newcommand{\sd}{\ssigma}
\newcommand{\vars}{\ssigma^2}
\newcommand{\charfun}{\sphi}
\newcommand{\vecX}{\bm{X}}

\makeatletter
\DeclareRobustCommand{\expe}{\mathsf{E}\@ifstar\@firstofone\@expe}
\newcommand{\@expe}[1]{\left[#1\right]}
\makeatother

\makeatletter
\DeclareRobustCommand{\var}{\mathsf{Var}\@ifstar\@firstofone\@var}
\newcommand{\@var}[1]{\left[#1\right]}
\makeatother

\begin{document}
\section{Statement}
Let $X$ be a continuous random variable whose probability density function is given by
\begin{equation}
\pdf_X(x) = \left\{ \begin{array}{l@{\quad}l} c\,, & -1 \le x \le 1, \\ 0\,, &  \textnormal{otherwise.} \end{array}\right.
\end{equation}
\begin{enumerate}
	\item What is the value of $c$?
	\item Derive the expression of the cumulative distribution function of $X$.
	\item Derive the expression of the characteristic function of $X$.
	\item Let $Y$ be the continuous random variable defined as $Y = 3 \mathrm{e}^X + 1$.
	Derive the expression of the probability density function of $Y$.
\end{enumerate}

\section{Solution}
\begin{enumerate}
	\item In order to find the value of $c$,
	we must recall that, by the definition of the probability density function,
	\begin{equation}
	\int_{-\infty}^{+\infty} \pdf_X(x) \dif x = 1\,.
	\end{equation}
	With this in mind, we find that
	\begin{equation}
	\int_{-1}^{1} c \dif x = 1\,.
	\end{equation}
	Solving the integral on the left yields
	\begin{equation}
	\Big[cx\Big]_{x = -1}^{x = 1} = 1 \iff \boxed{c = \frac{1}{2}}\,.
	\end{equation}
	\item The cumulative distribution function $\cdf_X(x)$ is defined as
	\begin{equation}
	\cdf_X(x) = \int_{-\infty}^{x} \pdf_X(\xi) \dif \xi \implies \boxed{\cdf_X(x) = \left\{ 
	\begin{array}{l@{\quad}l}
	0\,, & x < -1\,, \\
	\\
	(x+1)/2\,, & -1 \le x < 1\,, \\
	\\
	1\,, & x \ge 1\,.
	\end{array} \right.}
	\end{equation}
	\item The characteristic function $\charfun_X(q)$ is defined as
	\begin{equation}
	\charfun_X(q) = \expe{\mathrm{e}^{\imagj q X}} = \int_{-\infty}^{+\infty} \mathrm{e}^{\imagj q x} \pdf_X(x) \dif x\,.
	\end{equation}
	This means that in our case,
	\begin{equation}
	\charfun_X(q) = \int_{-1}^{1} \mathrm{e}^{\imagj q x} \frac{1}{2} \dif x \iff \charfun_X(q) = \frac{1}{2} \bigg[ \frac{1}{\imagj q} \mathrm{e}^{\imagj q x}\bigg]_{x = -1}^{x = 1}\,.
	\end{equation}
	Substituting the values of $x$ in the previous equation then yields
	\begin{equation}
	\charfun_X(q) = \frac{1}{\imagj 2 q} (\mathrm{e}^{\imagj q} - \mathrm{e}^{-\imagj q}) \implies \boxed{\charfun_X(q) = \frac{\sinh(\imagj q)}{\imagj q}}\,.
	\end{equation}
	\item Since the transformation $u \colon x \mapsto 3 \mathrm{e}^x + 1$ is strictly increasing, $u$ has an inverse, $u^{-1} \colon y \mapsto \ln\left(\frac{y-1}{3}\right)$.
	Thus, we have
	\begin{align}
	\cdf_Y(y) &= \Pr{u^{-1}\left(3 \mathrm{e}^X + 1\right) \le u^{-1}(y)}\\
	&= \Pr{X \le \ln\left(\frac{y-1}{3}\right)} \\
	&= \cdf_X\left( \ln\left(\frac{y-1}{3}\right) \right)\\
	&= \left\{ 
	\begin{array}{l@{\quad}l}
	0\,, & y < 3/\mathrm{e} + 1\,, \\
	\\
	\left.\left(\ln\left(\frac{y-1}{3}\right)+1\right)\right/2\,, & 3/\mathrm{e} + 1 \le  y < 3 \mathrm{e} + 1\,, \\
	\\
	1\,, & y \ge 3\mathrm{e}+1\,.
	\end{array} \right.
	\end{align}
	We then finally take the derivative of $\cdf_Y(y)$
	in order to find $\pdf_Y(y)$:
	\begin{equation}
	\boxed{\pdf_Y(y) = \left\{ \begin{array}{l@{\quad}l} \frac{1}{2(y - 1)}\,, & 3/\mathrm{e} + 1 \le  y < 3 \mathrm{e} + 1, \\\\ 0\,, &  \textnormal{otherwise.} \end{array}\right.}
	\end{equation}
\end{enumerate}
\end{document}